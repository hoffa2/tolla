%!TEX root = thesis.tex

%:-------------------------- Preamble -----------------------

% Three languages are supported, which will be reflected in the logo on the front page. Pass the appropriate language
% specified as a class option to uit-thesis. Passing any other languages supported by babel will result in the default
% language on the frontpage. If no language is passed, the default is selected.
%  - USenglish (default)
%  - norsk
%  - samin
% The frontpage comes in two variants, Master's thesis and PhD. Master is default, use classoption 'phd' for the PhD version.
\documentclass[USenglish]{uit-thesis}

% Lorem ipsum
\usepackage{lipsum}

\makeglossaries

% Add external glossaryentries
\loadglsentries{acronyms}
\newglossaryentry{thesis}
{
  name=thesis,
  description={is a document submitted in support of candidature for an
    academic degree or professional qualification presenting the author's
    research and findings
    },
}
\newglossaryentry{lage}
{
  name={long ass glossary entry},
  description={is a long ass entry with a lot of text describing the properties of the glossary entry. Hopefully this spans some lines now.
  },
}


\newcommand{\listdefinitionname}{My list of definitions}
\newlistof{definition}{def}{\listdefinitionname}
\newcommand{\definition}[1]{%
  \refstepcounter{definition}%
  \par\noindent\textbf{The Definition~\thedefinition. #1}%
  \addcontentsline{def}{definition}
    {\protect\numberline{\thechapter.\thedefinition}#1}\par%
}

\begin{document}

%:-------------------------- Frontpage ------------------------

\title{To be announces}
\subtitle{Subtitle}			% Optional
\author{Helge Hoff}
\thesisfaculty{Faculty of Science and Technology \\ Department of Computer Science}
\thesisprogramme{Master thesis in Computer Science … Month 20xx}
%\ThesisFrontpageImage{example_image.jpg}	% Optional

\maketitle

%:-------------------------- Frontmatter -----------------------
\frontmatter

\begin{epigraph}
\epigraphitem{}{Edsger Dijkstra}
\epigraphitem{Beware of bugs in the above code;\\I have only proved it correct, not tried it.}{Donald Knuth}
\end{epigraph}

\begin{abstract}
\lipsum[2-3]
\end{abstract}

\begin{acknowledgement}
\lipsum[4-8]
\end{acknowledgement}

\tableofcontents

\listofdefinition

%:-------------------------- Mainmatter -----------------------
\mainmatter

\chapter{Introduction}

\section{Problem Statement}


\begin{quote}
    Build a containerized system for acces
\end{quote}

\chapter{Background}
\section{Data privacy \& Personal Data}
Personal data is any information that can be used to identify to which individual it belongs.
By that, an identifiable individual is one can be identified by identifiers such as name,
location data, or online identifiers.

Data privacy (or information privacy) 
\section{General Data Protection Regulation}
The new \gls{gdpr} was approved by the European Parliament on April 14 2016, and will replace the current
data protection directive of 1995, in May 2018.
Its intention is to address the current state of data privacy by transfering the power from
service providers, namely application service providers and storage service providers,
to individuals of The European Union, regardless of jurisdiction.
By that, its aim is to protect all EU citezens from data-breaches leading to
the disclosure or privacy in an increasingly data-driven world, which stands as
contrary to the time in which the last directive was made.
Although many principles of data privacy still hold,
the new directive will enforce key changes to achieve the empowerment of end-users.
The conditions for consent have been strengthened to the extent of which
ambiquitus illegible terms and conditions are illegitimate and must be replaced
by incontrovertible forms of consent; and it shall be as easy to give as to
revoke consent.
The directive presents the notion of \textit{Right to Access} which encompasses
both the opportunity to access data belonging to you and retrieve the following information about
the processing of personal data:
\begin{itemize}
    \item if and where the data is being processed
    \item the purpose of processing
    \item the categories of data concerned
    \item to whom the personal data have been disclosed
    \item whether any automated decision-making is involved
\end{itemize}
Strongly related to the \textit{Right of Access} is the right for a data subject
retrieve or transfer the data provided in a \textit{\"machine readable format\"},
which adheres to the primitives in a data-driven context.

The control is further strengthened by enforcing \textit{Breach Notification} and
\textit{Right to be forgotten (Right to erasure)}.
The \textit{Right to be forgotten} is coupled with the conditions for consent,
in which the act of consenting to the collection data must state the time period
for which the data is to be stored, and at the end of which must be forgotten (Deleted).
\textit{Breach Notification} is mandatory

\textit{Data Privacy by Default} is at the core of the directive, and states
that: \textit{any data controller shall implement appropriate technical and organizational
measures for ensuring that, by default, only personal data which are necessary for each spesific
purpose of the processing are processed}, or simply data minimization.
This applies to both the retrieval and processing of data.
A controller collecting data shall only collect data which are necessary for the intent
of collecting that data.
In addition to that, \gls{gdpr} adds to the notion of data minimization by classifying
data having been collected for a purpose must fall under the \textit{Right to be Forgotten}
if the purpose changes.



All of these changes share the common objective of allowing citizens to
be in control of their data.
Failing to be compliant with any of the articles making up the
directive will result in being penilized with fines up to $ 4\% $ of a company's annual turnover
or 20 Million (whichever is greater), which can be substantial for most companies.
\section{Containers \& Microservices}


\chapter{Related work}


\end{document}

