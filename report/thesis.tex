        l%!TEX root = thesis.tex

%:-------------------------- Preamble -----------------------

% Three languages are supported, which will be reflected in the logo on the front page. Pass the appropriate language
% specified as a class option to uit-thesis. Passing any other languages supported by babel will result in the default
% language on the frontpage. If no language is passed, the default is selected.
%  - USenglish (default)
%  - norsk
%  - samin
% The frontpage comes in two variants, Master's thesis and PhD. Master is default, use classoption 'phd' for the PhD version.
\documentclass[USenglish]{uit-thesis}

\makeglossaries

% Add external glossaryentries
\loadglsentries{acronyms}
\newglossaryentry{thesis}
{
  name=thesis,
  description={is a document submitted in support of candidature for an
    academic degree or professional qualification presenting the author's
    research and findings
    },
}
\newglossaryentry{lage}
{
  name={long ass glossary entry},
  description={is a long ass entry with a lot of text describing the properties of the glossary entry. Hopefully this spans some lines now.
  },
}


\newcommand{\listdefinitionname}{My list of definitions}
\newlistof{definition}{def}{\listdefinitionname}
\newcommand{\definition}[1]{%
  \refstepcounter{definition}%
  \par\noindent\textbf{The Definition~\thedefinition. #1}%
  \addcontentsline{def}{definition}
    {\protect\numberline{\thechapter.\thedefinition}#1}\par%
}

\begin{document}

%:-------------------------- Frontpage ------------------------

\title{Tolla - towards GDPR compliancy}
\subtitle{Subtitle}			% Optional
\author{Helge Hoff}
\thesisfaculty{Faculty of Science and Technology \\ Department of Computer Science}
\thesisprogramme{Master thesis in Computer Science … Month 20xx}
%\ThesisFrontpageImage{example_image.jpg}	% Optional

\maketitle

%:-------------------------- Frontmatter -----------------------
\frontmatter

\begin{epigraph}
\epigraphitem{}{Edsger Dijkstra}
\epigraphitem{Beware of bugs in the above code;\\I have only proved it correct, not tried it.}{Donald Knuth}
\end{epigraph}

\begin{abstract}
\end{abstract}

\begin{acknowledgement}
\end{acknowledgement}

\tableofcontents

\printacronyms

\listofdefinition

%:-------------------------- Mainmatter -----------------------
\mainmatter

\chapter{Introduction}

\section{Problem Statement}


\begin{quote}
    Build a containerized system for acces
\end{quote}

\chapter{Background}
\section{Data privacy \& Personal Data}
Personal data is any information that can be used to identify to which individual it belongs.
By that, an identifiable individual is one can be identified by identifiers such as name,
location data, or online identifiers.

Data privacy (or information privacy)
\section{General Data Protection Regulation}
The new \gls{gdpr} was approved by the European Parliament on April 14 2016, and will replace the current
data protection directive of 1995, in May 2018.
Its intention is to address the current state of data privacy by transfering the power from
service providers, namely application service providers and storage service providers,
to individuals of The European Union, regardless of jurisdiction.
By that, its aim is to protect all EU citezens from data-breaches leading to
the disclosure or privacy in an increasingly data-driven world, which stands as
contrary to the time in which the last directive was made.
Although many principles of data privacy still hold,
the new directive will enforce key changes to achieve the empowerment of end-users.
The conditions for consent have been strengthened to the extent of which
ambiquitus illegible terms and conditions are illegitimate and must be replaced
by incontrovertible forms of consent; and it shall be as easy to give as to
revoke consent.
The directive presents the notion of \textit{Right to Access} which encompasses
both the opportunity to access data belonging to you and retrieve the following information about
the processing of personal data:
\begin{itemize}
    \item if and where the data is being processed
    \item the purpose of processing
    \item the categories of data concerned
    \item to whom the personal data have been disclosed
    \item whether any automated decision-making is involved
\end{itemize}
Strongly related to the \textit{Right of Access} is the right for a data subject
retrieve or transfer the data provided in a \textit{\"machine readable format\"},
which adheres to the primitives in a data-driven context.

The control is further strengthened by enforcing \textit{Breach Notification} and
\textit{Right to be forgotten (Right to erasure)}.
The \textit{Right to be forgotten} is coupled with the conditions for consent,
in which the act of consenting to the collection data must state the time period
for which the data is to be stored, and at the end of which must be forgotten (Deleted).
\textit{Breach Notification} is mandatory

\textit{Data Privacy by Default} is at the core of the directive, and states
that: \textit{any data controller shall implement appropriate technical and organizational
measures for ensuring that, by default, only personal data which are necessary for each spesific
purpose of the processing are processed}, or simply data minimization.
This applies to both the retrieval and processing of data.
A controller collecting data shall only collect data which are necessary for the intent
of collecting that data.
In addition to that, \gls{gdpr} adds to the notion of data minimization by classifying
data having been collected for a purpose must fall under the \textit{Right to be Forgotten}
if the purpose changes.

All of these changes share the common objective of allowing citizens to
be in control of their data.
Failing to be compliant with any of the articles making up the
directive will result in being penilized with fines up to $ 4\% $ of a company's annual turnover
or 20 Million (whichever is greater), which can be substantial for most companies.
\section{Containers \& Microservices}
Containers, in contrary to VM's, run directly on the host system, achieving virtualization
by using isolation primitives implemented in the linux kernel \cite{Soltesz:2007:COS:1272998.1273025, 7095802}.
In other words, containers sandbox processes from each other.
Container are mostly known as lighweight VMs, mainly becuase they take up less space
and can be regarded as isolated processes, and not needing to contain an entire Operating System.
When bulding Microservices containers have been widely used because they have properties
that are suitable for orchestrating, scaling, and decoupling several isolated processes into an application \cite{micro1}.

The most used implementation of linux containers are dockers.
Docker has a whole ecosystem for storing, deplying and building images.

Security-wise containers are arguably not as strong as VMs, one of the reasons being
that containers run with root priviliges, which imposes a potential attack vector.


\chapter{Related work}
\section{GDPR}
Not much have been done on \gls{gdpr} compliant implementations.
\cite{danishguy} presents a basic proof-of-concept \gls{gdpr}-compliant framework addressing
traceability, \textit{Right to Access} and \textit{Right to Erasure}.
To achieve these properties they attach meta-data (\textit{customer record}) to user identifiable data consisting of
policies that apply to the data, time of creation (to assert lifespan), and
whether the data is located at other controllers.
Conclusively, their main contribution is a service that fascilitates
the traceability of user data.

\section{Legacy Applications in Containers}


\chapter{Design}
\section{Isolation}
To contextualize the methods to which \textit{Privacy By design} can be realized, we
propose multiple solitions to data isolation, each of which having favourable characteristics.
Prior to that, there are two important requirements: prevent leaking (disclosing) data
from one user to other users, and the extent to which the solution can be
integrated with current infrastructure; that is, accomplish the level of isolation
without changing the logic of data storage services.
By that, we do not intend to create new ways of storing data (in a compliant manner),
rather investigate the possibilites of transparently integrate \gls{gdpr} compliancy
into data stores.

\subsection{Logical Seperation}
The first method of isolation is Logical space seperation, which is implemented
by many databases, both Relational and NOSQL.
Logical space seperation can be realized through database views, which
allow filtration of table content according to a certian criterion.
I.e in multi-tenancy queries can be issued along with a tenant id used by the database
to only disclose rows that are identified with the tenant id.
This is, however, an approach vulnerable to pragrammatical errors and does not fit our
multi-purpose solution of both having multi-tenancy and having the ability to do aggregates
on data.
In addition, it does not fit our goal of transparency either since using
database views would require putting the isolation logic to clients by writing spesialized
queries.

\subsection{Physical Seperation}
Physical seperation can be achieved by containerizing multiple databases where
a single container holds the information identified to one user.
The trade-offs here are inefficent utilization of space and resources compared to
the security properties achieved.
Running multiple containers (possibly thousands), one per user, saturates
the host on which they run.
Comparatively, having the seperation of user information by dockerization means
that the data volumes used by each database are inaccessible by other containers.

\subsection{Encryption}
Another way of lowering the risks of disclosing data, is to encrypt user information
and only allowing disclosure of that information (encryption) to processes that
are granted access to it.
There are multiple angles in which to create encryption based privacy.
Reference uses ID-based encryption that allows process to decrypt data if and only if
the process holds the information used to encrypt the data.
Another newly introduced scheme is \gls{byoe}, sometimes reffered to as \gls{byok},
where users provide their own public key-pair.
Doing so allows user data to be encrypted at all time, and allowing only authorized
processes to access data.

\subsection{Solution}
Each of these solutions have advantageous properties, however, supporting
legacy application whilst retaining the required properties with regards to security we deem
physical seperation as the direction in which to explore.
The encryption scheme will be infeasible with regards to encrypting the contents of database.
Having physical seperation, the threat of accidentally disclosing a user's information
to other users.
As depicted in, the data stores act as reference monitors that deny or accept request, and
audits access to data.
Therefore the solution we propose is a container-based data store with a granularity
of per-container-access.

\subsection{Assumptions on Architecture}
Before giving a description of the system, we make the following assumptions.
Having one containerized database per user we assume a unified schema accross
all users.
And we do not address the granularity or categories of data, meaning that
a process is only either denied or granted access to data based on intent and not data category.
The system consists of processes, each of which having a purpose for processing data.
All of these purposes will be valid for the system, that is,
the system is one single application and can not be divided into multiple applications
with different sets of intentions.

\section{Certificate-based Access Control}
Mapping the high-level intents of a process to which a user consents are realized by
leveraging the prevalent \gls{tls}.
Utilizing a security mechanism that is already a part of any architecure gives
the property of seamless intergration with legacy applications.

Using certificate extentensions has prevously been explored to implement role-based access control \cite{7095802, 1189190}.
We however, explore the use of certificate extensions to implement a intent-based access control.

The access control works by having a \gls{ca} delegating certificates in which the process'
intent is stored.
In that way, a \gls{tls} connection is directly coupled with an intent and can be
handled accordingly.
I.e if a user consents to his data being used for calculating sleep patterns, a \gls{tls}
connection from a process having that as its intent will be granted access to the user's
container; and if not processes having other intents will not have access to the user's container.
This also fascilitates the compliance with \gls{gdpr} artice SOMETHING that all access and reasons
for proccessing shall be logged such that the users are informed about what happends to their data.

\begin{lstlisting}[frame=single, caption={Intent certificate. Some fields are left out for brievity}, label={lst:certificate}]
    Data:
  Version:  v3
  Serial Number: 0x1
    Signature Algorithm: sha256WithRSAEncryption
  Issuer: CN=localhost, ST=TR, C=NO
  Validity:
     Not Before: Nov  1 18:01:29 2017 GMT
     Not After : Nov 11 18:01:29 2017 GMT
  Subject: CN=localhost, ST=TR, C=NO
  Extensions:
    Intent: Intention for processing
\end{lstlisting}

An example certificate of a processor is shown in listing \ref{lst:certificate}.
The intent for processing is expressed as a certificate extension, where the
intent can be of an arbitrary string which the \gls{ca} verifies.
All processes in the system registers their intentions which are divided into two
groups, collection and reading, respectively.
Since our system can be viewed as a single application, all processes' intents
are regigestered with the system and will be puproses to which users must consent.

\section{Containerizing legacy applications}
Conatinerizing legacy applications have been done to migrate
such applications to the cloud \cite{7092950}.
Integrating the access control scheme with legacy applications is done by
building base-images from which applications are containerized.

Containerizing a client (or processor) to fit system is done transparently upon
initialization.
Intents, along with meta-data bout the process, are written along with the building procedures
of the application (Dockerfile).
When a container starts running, a built in process from the base-image creates a certificate request
from the information provided in the \textit{Dockerfile} and sent to the \gls{ca} to be signed.
The end result is that three files are written to a directory: the public key-pair,
the certificate of the \gls{ca} and the proccess' own certificate.
Therefore, the only alterations to the application that is requires is from where
these three files are read from.
Thereby, no alternations to application logic related to recieving and writing data have
to be made.

\section{Architecture}

\subsection{Data stores}
A data store, as mentioned earlier, is a container running an arbitrary data store
with additional security logic.
To minmize the contaner storage overhead we leverage a built-in docker feature.
Docker calls this feature \textit{the image layer} and is a read-only layer of application code.
This feature is a part of the image infrastructure of docker.
An image is a lightweight, stand-alone, executable packages that includes everything
needed to run a piece of software, and contains files relevent to the runtime of the software:
code, dependencies, environmental variables, and config files.
Whilst containers can be described as an instance of an image.
A container on the other hand is a runtime instance of an image, completely isolated from the host environment, reading all software logic from the read-only layer,
whilst all writes/updates are done to a writable layer spesific and private to that container.
Therefor, docker makes it possible to run multiple containers, all having a physical size of zero upon start,
without duplicating space (images).
The virtual space however is the size of the underlying docker image.
The only disk usage will therfore be determined by how much is written to it.


\subsection{Logging}
Artice 15 of the \gls{gdpr} says that users have right to obtain information
about whether personal data concerning him or her are being processed.
To fulfill that requirement all access to the data stores are gathered continously
from all containers and appended to a log.
I.e, if a process with a purpose A access the data of a user, it will be logged to
the centralized log that is visible to users.

\subsection{Container orchestrations}
Docker exposes a remote client API accessible if configured.
The \gls{api} allows the same operations as Docker's CLI application.
With that capability we have created a orchestraion process managing the
security and orchestration of the data stores.
When a user wishes to register with the system the orchestration tool
adds a container to the network from a shared image.
The container is identified with that particular user and has full control over
for which purposes the data is used, meaning that consents can be revoked or the
container deleted at any time.

The containers run in a sub-net which imposes a restriction that application containers
can only access a data store if they reside in the same sub-net.
The orchestration process also implements what we call a \textit{Consent engine}.
The \textit{Consent engine} stores all logic that maps users to containers,
containers to consents, and based the those determines access policies.
I.e if we have consents $ [A, B, C] $ which are directly related to the purposes
for which applications do processing.
Meantion that all applications purposes are logged into the system and all those are
to be consented. And the possibility of having ceveral consent engines to orchestrate
multiple systems having different sets of purposes.

\subsection{GDPR compliancy}

\section{Datastores}

\bibliography{citations}
\bibliographystyle{ieeetr}


\end{document}

